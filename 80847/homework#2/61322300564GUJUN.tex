\documentclass{article}
\usepackage[utf8]{inputenc}
\usepackage{geometry}
\usepackage{amsmath}
\usepackage{url}
\usepackage[utopia]{mathdesign}
\usepackage{lipsum}  
\usepackage{lmodern}
\usepackage{listings}

 \geometry{
 a4paper,
 total={170mm,257mm},
 left=20mm,
 top=20mm,
 }
 \usepackage{graphicx}
 \usepackage{titling}

 \title{Report \#2}
\author{GU JUN}
\date{November 1, 2024}
 
 \usepackage{fancyhdr}
\fancypagestyle{plain}{%  the preset of fancyhdr 
    \fancyhf{} % clear all header and footer fields
%    \fancyfoot[R]{\includegraphics[width=5cm]{}}
    \fancyfoot[L]{\today}
    \fancyhead[L]{Analytical Mechanics 80848\#1}
    \fancyhead[R]{HIRAI SHINCHI}
}
\makeatletter
\renewcommand{\maketitle}{%
  \newpage
  \null
  \vskip 1em%
  \begin{center}%
  \let \footnote \thanks
    {\LARGE \@title \par}%
    \vskip 1em%
    %{\large \@date}%
  \end{center}%
  \par
  \vskip 1em}
\makeatother


\begin{document}

\maketitle

\noindent\begin{tabular}{@{}ll}
    Student & \theauthor\\
    ID number & 6132230056 \\
\end{tabular}

\section*{Problem Statement}
Assume that a system is described by four coordinates $q_1$ through $q_4$.
Two constraints R1 and R2 are imposed on the system.
Let $\bf{q} = \left[q_1, q_2, q_3, q_4\right]^{\top}$ 
and $\bf{R} = \left[R_1, R_2\right]^{\top}$.
Let $\bf{g}_1$ and $\bf{H}_1$ be gradient vector
and Hessian matrix related to $R_1$ while $g_2$ and $H_2$ be gradient vector
and Hessian matrix related to $R_2$. Let $J$ be Jacobian given by
\begin{equation}
  J=\left[\begin{array}{llll}
  \partial R_1 / \partial q_1 & \partial R_1 / \partial q_2 & \partial R_1 / \partial q_3 & \partial R_1 / \partial q_4 \\
  \partial R_2 / \partial q_1 & \partial R_2 / \partial q_2 & \partial R_2 / \partial q_3 & \partial R_2 / \partial q_4
  \end{array}\right]
\end{equation}
Show the following equations:
\begin{equation}
  \begin{aligned}
  \dot{\boldsymbol{R}} & =J \dot{\boldsymbol{q}} \\
  \ddot{\boldsymbol{R}} & =J \ddot{\boldsymbol{q}}+\left[\begin{array}{c}
  \dot{\boldsymbol{q}}^{\top} H_1 \dot{\boldsymbol{q}} \\
  \dot{\boldsymbol{q}}^{\top} H_2 \dot{\boldsymbol{q}}
  \end{array}\right]
  \end{aligned}
  \end{equation}

% Solution
\section*{Solution}
\subsection*{1. $\dot{\bf{R}} = J \dot{q}$}
\begin{equation}
  \dot{R}=\frac{\partial R}{\partial q} \dot{q}=J \dot{q}
\end{equation}
Here, we have
\begin{equation}
  J = \frac{\partial R}{\partial q} 
\end{equation}
So, the equation $\dot{\bf{\bf{R}}} = J \dot{q}$ is proved.
\subsection*{2. $\ddot{\bf{R}} = J \ddot{q} + \begin{bmatrix} \dot{q}^T H_1 \dot{q} \\ \dot{q}^T H_2 \dot{q} \end{bmatrix}$ }
Differentiating the equation $\dot{\bf{R}} = J \dot{q}$ with respect to time, we have
\begin{equation}
\ddot{\bf{R}} = \frac{d}{dt}(J \dot{q}) = \frac{dJ}{dt} \dot{q} + J \ddot{q}
\end{equation}
The time derivative of $J$ can be expanded using the gradient and Hessian matrices:
\begin{equation}
\ddot{\bf{R}} = \frac{d}{dt}(J \dot{q}) = \frac{dJ}{dt} \dot{q} + J \ddot{q}
\end{equation}
Finally, we have
\begin{equation}
\ddot{\bf{R}} = J \ddot{q} + \begin{bmatrix} \dot{q}^T H_1 \dot{q} \\ \dot{q}^T H_2 \dot{q} \end{bmatrix}
\end{equation}
\end{document}
