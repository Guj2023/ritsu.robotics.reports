\documentclass{article}
\usepackage[utf8]{inputenc}
\usepackage{geometry}
\usepackage{amsmath}
\usepackage{url}
\usepackage[utopia]{mathdesign}
\usepackage{lipsum}  
\usepackage{lmodern}
\usepackage{listings}
\usepackage{tikz}
\usetikzlibrary{arrows.meta, positioning}

 \geometry{
 a4paper,
 total={170mm,257mm},
 left=20mm,
 top=20mm,
 }
 \usepackage{graphicx}
 \usepackage{titling}

 \title{Report \#5}
\author{GU JUN}
 
 \usepackage{fancyhdr}
\fancypagestyle{plain}{%  the preset of fancyhdr 
    \fancyhf{} % clear all header and footer fields
%    \fancyfoot[R]{\includegraphics[width=5cm]{}}
    \fancyfoot[L]{\today}
    \fancyhead[L]{Analytical Mechanics 80848\#5}
    \fancyhead[R]{HIRAI SHINCHI}
}
\makeatletter
\renewcommand{\maketitle}{%
  \newpage
  \null
  \vskip 1em%
  \begin{center}%
  \let \footnote \thanks
    {\LARGE \@title \par}%
    \vskip 1em%
    %{\large \@date}%
  \end{center}%
  \par
  \vskip 1em}
\makeatother


\begin{document}

\maketitle

\noindent\begin{tabular}{@{}ll}
    Student & \theauthor\\
    ID number & 6132230056 \\
\end{tabular}

\section{Show that $R(\boldsymbol{q})$ orthogonal.}
To show that  $R(q)$, the rotation matrix derived from a quaternion $q$ , is orthogonal, we need to verify that:

\begin{equation}
  R(q)^\top R(q) = I
\end{equation}

where  $R(q)^\top$  is the transpose of  $R(q)$ , and  $I$  is the identity matrix.


\begin{equation}
  R(q) = \begin{bmatrix}
    1 - 2(y^2 + z^2) & 2(xy - wz) & 2(xz + wy) \\
    2(xy + wz) & 1 - 2(x^2 + z^2) & 2(yz - wx) \\
    2(xz - wy) & 2(yz + wx) & 1 - 2(x^2 + y^2)
    \end{bmatrix}    
\end{equation}
where  $w$, $x$, $y$, $z$  are the components of the quaternion.

It's transpose is:
\begin{equation}
  R(q)^\top = \begin{bmatrix}
    1 - 2(y^2 + z^2) & 2(xy + wz) & 2(xz - wy) \\
    2(xy - wz) & 1 - 2(x^2 + z^2) & 2(yz + wx) \\
    2(xz + wy) & 2(yz - wx) & 1 - 2(x^2 + y^2)
    \end{bmatrix}
\end{equation}

Now, we multiply  $R(q)^\top$  by  $R(q)$ :
\begin{equation}
  R(q)^\top R(q) =
    \begin{bmatrix}
      1 & 0 & 0 \\
      0 & 1 & 0 \\
      0 & 0 & 1
    \end{bmatrix}
  \end{equation}
The result of  $R(q)^\top R(q)$  is the identity matrix  $I$ :

$R(q)^\top R(q) = I$.

So the rotation matrix  $R(q)$  is orthogonal.
\section{Show $\dot{A}\boldsymbol{q}=A\dot{\boldsymbol{q}}$, $\dot{B}\boldsymbol{q}=B\dot{\boldsymbol{q}}$, and $\dot{C}\boldsymbol{q}=C\dot{\boldsymbol{q}}$.}

\begin{equation}
A = \left[\begin{array}{cccc}
  q_0 & q_1 & -q_2 & -q_3 \\
  q_3 & q_2 & q_1 & q_0 \\
  -q_2 & q_3 & -q_0 & q_1
  \end{array}\right], \quad
  B = \left[\begin{array}{cccc}
    -q_3 & q_2 & q_1 & -q_0 \\
    q_0 & -q_1 & q_2 & -q_3 \\
    q_1 & q_0 & q_3 & q_2
    \end{array}\right], \quad
  C = \left[\begin{array}{cccc}
    q_2 & q_3 & q_0 & q_1 \\
    -q_1 & -q_0 & q_3 & q_2 \\
    q_0 & -q_1 & -q_2 & q_3
    \end{array}\right].
\end{equation}

Each element of  $A$  depends on  $q_0$, $q_1$, $q_2$, $q_3$ . Taking the derivative with respect to time:

\begin{equation}
  \dot{A} = \begin{bmatrix}
    \dot{q}_0 & \dot{q}_1 & -\dot{q}_2 & -\dot{q}_3 \\
    \dot{q}_3 & \dot{q}_2 & \dot{q}_1 & \dot{q}_0 \\
    -\dot{q}_2 & \dot{q}_3 & -\dot{q}_0 & \dot{q}_1
    \end{bmatrix}.
\end{equation}

When $\dot{A}$  is multiplied by  $\boldsymbol{q}$ :

\begin{equation}
  \dot{A}\boldsymbol{q} = 
    \begin{bmatrix}
    \dot{q}_0q_0 + \dot{q}_1q_1 - \dot{q}_2q_2 - \dot{q}_3q_3 \\
    \dot{q}_3q_0 + \dot{q}_2q_1 + \dot{q}_1q_2 + \dot{q}_0q_3 \\
    -\dot{q}_2q_0 + \dot{q}_3q_1 - \dot{q}_0q_2 + \dot{q}_1q_3
    \end{bmatrix}    
\end{equation}

Similarly, consider  $A\dot{\boldsymbol{q}}$ , where:

\begin{equation}
  A\dot{\boldsymbol{q}} = \begin{bmatrix}
    q_0 & q_1 & -q_2 & -q_3 \\
    q_3 & q_2 & q_1 & q_0 \\
    -q_2 & q_3 & -q_0 & q_1
    \end{bmatrix}
    \begin{bmatrix}
    \dot{q}_0 \\
    \dot{q}_1 \\
    \dot{q}_2 \\
    \dot{q}_3
    \end{bmatrix}    
\end{equation}

Performing the multiplication:

\begin{equation}
  A\dot{\boldsymbol{q}} = \begin{bmatrix}
    q_0\dot{q}_0 + q_1\dot{q}_1 - q_2\dot{q}_2 - q_3\dot{q}_3 \\
    q_3\dot{q}_0 + q_2\dot{q}_1 + q_1\dot{q}_2 + q_0\dot{q}_3 \\
    -q_2\dot{q}_0 + q_3\dot{q}_1 - q_0\dot{q}_2 + q_1\dot{q}_3
    \end{bmatrix}    
\end{equation}
By inspecting the expressions for $\dot{A}\boldsymbol{q}$  and  $A\dot{\boldsymbol{q}}$ , we see that the two results are identical:

\begin{equation}
  \dot{A}\boldsymbol{q} = A\dot{\boldsymbol{q}}
\end{equation}.
Following the same approach, we can compute the derivatives of  $B$  and  $C$ , and we can proof:

\begin{equation}
  \dot{B}\boldsymbol{q} = B\dot{\boldsymbol{q}}, \quad \dot{C}\boldsymbol{q} = C\dot{\boldsymbol{q}}.

\end{equation}


\section{Show $\dot{A}\boldsymbol{q}=A\dot{\boldsymbol{q}}$, $\dot{B}\boldsymbol{q}=B\dot{\boldsymbol{q}}$, and $\dot{C}\boldsymbol{q}=C\dot{\boldsymbol{q}}$.}

\section{Show $\dot{H}\dot{\boldsymbol{q}}=0$}

\section{Show $H\dot{\boldsymbol{q}} = -\dot{H}\boldsymbol{q}$ and $\omega = -2\dot{H}\boldsymbol{q}$}

\section{Show $H H^{\top}=I_{3 \times 3}$}

\section{$H \dot{\boldsymbol{q}}=\dot{\boldsymbol{q}} \text { and } \dot{\boldsymbol{q}}=(1 / 2) H^{\top} \boldsymbol{\omega}$}

\end{document}
